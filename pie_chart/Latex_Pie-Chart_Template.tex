% Pie Template
\documentclass[ngerman,pdftex,12pt,a4paper]{scrreprt}

\usepackage[utf8]{inputenc} 	% ermöglicht die direkte Eingabe von Umlauten
\usepackage[T1]{fontenc} 		% Ausgabe aller zeichen in einer T1-Codierung (wichtig für die Ausgabe von Umlauten!)
\usepackage[ngerman]{babel} 	% deutsche Trennungsregeln und Übersetzung der festcodierten Überschriften

\usepackage{xcolor} 			% einfache Verwendung von Farben in nahezu allen Farbmodellen
\usepackage{pgf-pie} %ftp://ftp.fu-berlin.de/tex/CTAN/graphics/pgf/contrib/pgf-pie/pgf-pie-manual.pdf

\begin{document}

\chapter{Vorlage für Kreisdiagramm}

\begin{figure}[h]
\centering
\tikz{\pie [explode = {0, 0},
  color = {red!40, green!40} ] {
  61.4/ rot,
  38.6/ grün}}
\caption{Kuchendiagramm }
\label{fig:redgreen} 
\end{figure}


\end{document}