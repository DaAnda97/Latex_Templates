\documentclass[ngerman,pdftex,12pt,a4paper]{scrreprt}


%----------------------------------- Einbindung verwendeter Pakete ------------------------------------
\usepackage[utf8]{inputenc} 	% ermöglicht die direkte Eingabe von Umlauten
\usepackage[T1]{fontenc} 		% Ausgabe aller zeichen in einer T1-Codierung (wichtig für die Ausgabe von Umlauten!)
\usepackage[ngerman]{babel} 	% deutsche Trennungsregeln und Übersetzung der festcodierten Überschriften

\usepackage{graphicx} % Für Bilder
\usepackage{array}    % Extending the array and tabular environments 

% Tabellenformatierung
\newcolumntype{C}[1]{>{\centering\arraybackslash}p{#1}} % Für zentrierte Zellen mit Zeilenumbruch
\setlength\extrarowheight{3pt} % Erhöhung des Tabellenabstands nach oben
\newcommand{\thickhline}{\noalign{\hrule height 1.5pt}} % Dicke horizontale Linie
% Für dicke vertikale Linie: !{\vrule width 2pt} anstatt |


\begin{document}

\chapter{Bild}

\begin{figure}[h]
\centering
\includegraphics[width=0.8\textwidth]{Airplane.jpg} %Pfad zum Bild
\caption{Flugzeug}
\label{fig:eBanking} 
\end{figure}

\chapter{Tabelle}

Dank der richtigen Formatierung ist der Text mittig und es können sowohl dicke horizontale als auch vertikale Linien verwendet werden.

\begin{table}[h] 
\caption{OWASP Top 10 - 2013 \label{tab:topten2013}}
\begin{tabular}{|c !{\vrule width 2pt} C{6cm}|C{7cm}|}\hline
   \textbf{Ranking} & \textbf{Szenario} & \textbf{Beispiel/ Bedrohung} \\ \thickhline
   1 & Injection & Zugriff auf die Datenbank \\ \hline 
   2 & Fehler in Authentifizierung und Session - Management) & Eindringen in fremde Benutzerkonten \\ \hline
   3 & Cross-Site Scripting (XSS) & Eindringen in fremde Benutzerkonten  \\ \hline
   4 & Unsichere direkte Objektreferenzen & Zugriff auf geschützte oder fremde Daten \\ \hline
   5 & Sicherheitsrelevante Fehlkonfiguration & Unsichere Konfiguration der eingesetzten Systeme \\ \hline
   6 & Verlust der Vertraulichkeit sensibler Daten & Eindringen in fremde Benutzerkonten \\ \hline
   7 & Fehlerhafte Autorisierung auf Anwendungsebene & Unvollständige Überprüfung der Zugriffsberechtigung von Funktionen \\ \hline
   8 & Cross-Site Request Forgery (CSRF) & Im Browser des Benutzers unbemerkt Aktionen
ausführen \\ \hline
   9 & Nutzung von Komponenten mit bekannten Schwachstellen & Zugriff auf geschützte oder fremde Daten \\ \hline
   10 & Ungeprüfte Um- und Weiterleitungen & Fehlerhafte Zielbestimmung der Weiterleitung \\ \hline
\end{tabular}
\end{table}

\end{document}

