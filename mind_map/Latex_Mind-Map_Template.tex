% Mindmap Template
\documentclass[ngerman,pdftex,12pt,a4paper]{scrreprt}

\usepackage[utf8]{inputenc} 	% ermöglicht die direkte Eingabe von Umlauten
\usepackage[T1]{fontenc} 		% Ausgabe aller zeichen in einer T1-Codierung (wichtig für die Ausgabe von Umlauten!)
\usepackage[ngerman]{babel} 	% deutsche Trennungsregeln und Übersetzung der festcodierten Überschriften

\usepackage{xcolor} 			% einfache Verwendung von Farben in nahezu allen Farbmodellen
\usepackage{tikz}
\usetikzlibrary{mindmap,fit,arrows,calc,positioning} % Für die Zeichnungen

\begin{document}

\chapter{Vorlage für Mind-Maps}

\begin{figure}[h]
\begin{tikzpicture}
% http://tex.stackexchange.com/questions/46007/tikz-node-placement-and-arrow-drawing
% http://tex.stackexchange.com/questions/121809/positioning-text-in-tikz-drawings
% http://tex.stackexchange.com/questions/25681/tikz-what-affects-the-space-between-nodes-placed-with-below-of-syntax

\tikzstyle{box} = [rectangle, draw, minimum width=10em, minimum height=5.5ex, draw=black]
\tikzstyle{arrow} = [draw, latex-latex , thick]

\node [box, fill=blue, minimum width=18em] (one) {Eins};
\node [box, fill=blue, minimum width=18em, right=1.5 of one] (two) {zwei};
\coordinate (middle1) at ($(one.south)!0.5!(two.south)$);

\node [box, fill=green, below=1.5cm of middle1, minimum width=30em] (three) {Drei};
\node [box, fill=green, below=1.5cm of three, minimum width=30em] (four) {vier};
\coordinate [below=2cm of four] (middle2);

\node [box, fill=yellow, left=1.5cm of middle2] (five) {Fünf};
\node [box, fill=yellow, right=1.5cm of middle2] (six) {Sechs};

\path [arrow]  %One <-> Three
(one) 
	edge [left] node[xshift=-1em, fill=none, pos=.5, text height=.2em] {Von 1 nach 3}
(three);
\path [arrow] %Two <-> Three
(two) 
	edge [right] node[xshift=1em, fill=none, pos=.5, text height=.2em] {Von 2 nach 3}
(three);
\path [arrow]  %Three <-> Four
(three) 
	edge node[fill=white, pos=.5, text height=.2em] {Von 3 nach 4}
(four);

\path [arrow]  %Four <-> Five
(four) 
	edge [left] node[fill=none, pos=.5, text height=.2em] {Von 4 nach 5}
(five);
\path [arrow]  %Four <-> Six
(four) 
	edge [right] node[fill=none, pos=.5, text height=.2em] {Von 4 nach 6}
(six);
\path [arrow]  %Five <-> Six
(five) 
	edge node[fill=none] {}
(six);

\node [box, xshift=-4cm, fill=none, below=1.5cm of five, minimum height=4ex, minimum width=0em, draw=none] (legende) {Legende:};
\node [box, fill=blue, right=.3cm of legende, minimum height=4ex] (extSys) {Grün};
\node [box, fill=green, right=.3cm of extSys, minimum height=4ex] (eBP) {Grün};
\node [box, fill=yellow, right=.3cm of eBP, minimum height=4ex] (SysU) {Gelb};

\end{tikzpicture}
\caption{Zahlen}
\label{fig:numbers} 
\end{figure}


\end{document}